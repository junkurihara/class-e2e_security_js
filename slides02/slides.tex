%%%%%%%%%%%%%%%%%%%%%%%%%%%%%%%%%%%%%%%%%%%%%%%%%%%%%%%%%%%%%%%%%%%%%%%%%%%%%%%%%%%%%%%%%%%%%%%%%%%
%%%%%%%%%%%%%%%%%%%%%%%%%%%%%%%%%%%%%%%%%%%%%%%%%%%%%%%%%%%%%%%%%%%%%%%%%%%%%%%%%%%%%%%%%%%%%%%%%%%
%%%%%%%%%%%%%%%%%%%%%%%%%%%%%%%%%%%%%%%%%%%%%%%%%%%%%%%%%%%%%%%%%%%%%%%%%%%%%%%%%%%%%%%%%%%%%%%%%%%
\documentclass[12pt,dvipdfmx]{beamer}
%%%%%%%%%%%%%%%%%%%%%%%%%%%%%%%%%%%%%%%%%%%%%%%%%%%%%%%%%%%%%%%%%%%%%%%%%%%%%%%%%%%%%%%%%%%%%%%%%%%%%%
% pdfの栞・プロパティの字化けを防ぐ
\usepackage{atbegshi}
%\AtBeginShipoutFirst{\special{pdf:tounicode 90ms-RKSJ-UCS2}} %Windows
\AtBeginShipoutFirst{\special{pdf:tounicode EUC-UCS2}} %Linux, Mac
\usepackage{hyperref}
%%%%%%%%%%%%%%%%%%%%%%%%%%%%%%%%%%%%%%%%%%%%%%%%%%%%%%%%%%%%%%%%%%%%%%%%%%%%%%%%%%%%%%%%%%%%%%%%%%%%%%
%%%
%%% テーマの指定、省略時は default になる
%%%

 % フレームの指定、省略可
%%%%%%%%%%%%%%%%%%%%%%%%%%%% THEME
  %\usetheme{AnnArbor}
  %\usetheme{Antibes}
  %\usetheme{Bergen}
  %\usetheme{Berkeley}
  %\usetheme{Berlin}
  \usetheme{Boadilla}
  %\usetheme{boxes}
  %\usetheme{CambridgeUS}
  %\usetheme{Copenhagen}
  %\usetheme{Darmstadt}
  %\usetheme{default}
  %\usetheme{Dresden}
  %\usetheme{Frankfurt}
  %\usetheme{Goettingen}
  %\usetheme{Hannover}
  %\usetheme{Ilmenau}
  %\usetheme{JuanLesPins}
  %\usetheme{Luebeck}
  %\usetheme{Madrid}
  %\usetheme{Malmoe}
  %\usetheme{Marburg}
  %\usetheme{Montpellier}
  %\usetheme{PaloAlto}
  %\usetheme{Pittsburgh}
  %\usetheme{Rochester}
  %\usetheme{Singapore}
  %\usetheme{Szeged}
  %\usetheme{Warsaw}

% 省略可
%%%%%%%%%%%%%%%%%%%%%%%%%%%% COLOR THEME
  %\usecolortheme{albatross}
  %\usecolortheme{beetle}
  %\usecolortheme{crane}
  %\usecolortheme{default}
  %\usecolortheme{dolphin}
  %\usecolortheme{dove}
  %\usecolortheme{fly}
  %\usecolortheme{lily}
  %\usecolortheme{orchid}
  %\usecolortheme{rose}
  %\usecolortheme{seagull}
  %\usecolortheme{seahorse}
  %\usecolortheme{sidebartab}
  %\usecolortheme{structure}
  %\usecolortheme{whale}

% ヘッダ、フッタ、フレーム等を指定、省略可
  %%%%%%%%%%%%%%%%%%%%%%%%%%%% OUTER THEME
  %\useoutertheme{default}
  %\useoutertheme{infolines}
  %\useoutertheme{miniframes}
  %\useoutertheme{shadow}
  %\useoutertheme{sidebar}
  %\useoutertheme{smoothbars}
  %\useoutertheme{smoothtree}
  %\useoutertheme{split}
  %\useoutertheme{tree}

% タイトル、section, itemize/enumerate 環境、
% theorem 環境、図, 参考文献などのスタイルを指定、
% 省略可
  %%%%%%%%%%%%%%%%%%%%%%%%%%%% INNER THEME
  %\useinnertheme{circles}
  %\useinnertheme{default}
  %\useinnertheme{inmargin}
  \useinnertheme{rectangles}
  %\useinnertheme{rounded}


%\usefonttheme{}	% 省略可
%\logo{}		% 省略可

%%%%%%%%%%%%%%%%%%%%%%%%%%%%%%%%%%%%%%%%%%%%%%%%%%%%%%%%%%%%%%%%%%%%%%%%%%%%%%%%%%%%%%%%%%%%%%%%%%%
%%%%%%%%%%%%%%%%%%%%%%%%%%%%%%%%%%%%%%%%%%%%%%%%%%%%%%%%%%%%%%%%%%%%%%%%%%%%%%%%%%%%%%%%%%%%%%%%%%%
%%%%%%%%%%%%%%%%%%%%%%%%%%%%%%%%%%%%%%%%%%%%%%%%%%%%%%%%%%%%%%%%%%%%%%%%%%%%%%%%%%%%%%%%%%%%%%%%%%%
% navi. symbolsは目立たないが,dvipdfmxを使うと機能しないので非表示に
\setbeamertemplate{navigation symbols}{}

% 各種パッケージ
\usepackage{graphicx}
%\usepackage{url,cite}
\usepackage{amsmath}
\usepackage{amsthm} \theoremstyle{definition} %theorem環境が斜体になるので注意
\usepackage{amssymb} % AMS-TeX
\usepackage{setspace}

% \AtBeginSection[] % Do nothing for \section*
% { \begin{frame}<beamer> \frametitle{}
%    \tableofcontents[currentsection,subsectionstyle=hide]
%  \end{frame} } 

%appendixをページカウントしない
\newcommand{\backupbegin}{
   \newcounter{framenumberappendix}
   \setcounter{framenumberappendix}{\value{framenumber}}
}
\newcommand{\backupend}{
   \addtocounter{framenumberappendix}{-\value{framenumber}}
   \addtocounter{framenumber}{\value{framenumberappendix}} 
}

%%%%%%%%%%%%%%%%%%%%%%%%%%%%%%%%%%%%%%%%%%%%%%%%%%%%%%%%%%%%%%%%%%%%%%%%%%%%%%%%%%%%%%%%%%%%%%%%%%%%%%
% 本文・数式フォント
%\usepackage{palatino,mathpazo}
%\usepackage{times,mathptmx}
\usepackage[varg]{txfonts}
%\usepackage[varg]{pxfonts}

% \mathcal(\cal)の扱い
%\DeclareMathAlphabet{\mathcal}{OMS}{cmsy}{m}{n} %computer modern
%\DeclareMathAlphabet{\mathcal}{OMS}{txsy}{m}{n} %txfont
%\usepackage[psamsfonts]{eucal} % euler

% mathptmx時に数式モードのvをtxfontから借りる
% \DeclareSymbolFont{lettersA}{U}{txmia}{m}{it}
% \SetSymbolFont{lettersA}{bold}{U}{txmia}{bx}{it}
% \DeclareFontSubstitution{U}{txmia}{m}{it}
% \DeclareMathSymbol{v}{\mathalpha}{lettersA}{"33} %"


%上線 widebar, Widebar
\usepackage{accents}
\makeatletter
\def\widebar{\accentset{{\cc@style\underline{\mskip11mu}}}}
\makeatother


%%%%%%%%%%%%%%%%%%%%%%%%%%%%%%%%%%%%%%%%%%%%%%%%%%%%%%%%%%%%%%%%%%%%%%%%%%%%%%%%%%%%%%%%%%%%%%%%%%%%%%

% 定理環境
% \newtheorem{theorem}{Theorem}
% \newtheorem{lemma}[theorem]{Lemma}
% \newtheorem{corollary}[theorem]{Corollary}
% \newtheorem{definition}[theorem]{Definition}
% \newtheorem{example}[theorem]{Example}
\newtheorem{proposition}{Proposition}
\newtheorem{remark}{Remark}

%%%%%%%%%%%%%%%%%%%%%%%%%%%%%%%%%%%%%%%%%%%%%%%%%%%%%%%%%%%%%%%%%%%%%%%%%%%%%%%%%%%%%%%%%%%%%%%%%%%%%%
% 各種コマンド定義等
\def\Fig#1{Fig.\@\ref{#1}}
\def\Table#1{Table~\ref{#1}}
\def\Eq#1{Eq.\@(\ref{#1})}
\def\Eqs#1{Eqs.\@(\ref{#1})}
\def\Thm#1{Theorem~\ref{#1}}
\def\Lma#1{Lemma~\ref{#1}}
\def\Sect#1{Section~\ref{#1}}
\def\Rmk#1{Remark~\ref{#1}}
\def\Prop#1{Proposition~\ref{#1}}
\def\Coro#1{Corollary~\ref{#1}}
\def\Def#1{Definition~\@\ref{#1}}
\def\Prob#1{Problem~\@\ref{#1}}
\def\ie{{i.\@e.\@,~}}
\def\eg{{e.\@g.\@,~}}
\def\etal{{et al.}}

% 数式環境用
\def\rank{\mathsf{rank}\, }
\def\dim{\mathsf{dim}\, }
\def\rspace{\mathsf{span}}
\def\supp{\mathsf{supp}}
%\def\vec#1{\mathbf{#1}}
\def\F{\mathbb{F}}
\def\wt{\mathsf{wt}}
\def\c{\mathcal{C}}
\def\dc{\mathcal{C}^{\perp}}
\def\d{\mathcal{D}}
\def\dd{\mathcal{D}^{\perp}}
\def\g{\mathcal{G}}
\def\dg{\mathcal{G}^{\perp}}
\def\p{\mathcal{P}}
% \def\rspace{\mathsf{span}}
\def\supp{\mathsf{supp}}
\def\ker{\mathsf{Ker\ }}

%\def\bari#1{\{\widebar{#1}\}}
\def\bari#1{\,\overline{{\!\{#1\}\!}}\,}
%\def\bari#1{\bar{\{#1\}}}
\def\vecxi{Z_{\bari{i}}}
%\def\vecsxi{\vec{z}_i}
\def\tvector{X}
\def\tpackets{X_1,\dots,X_n}
\def\mvector{S}
\def\mpackets{S_1,\dots,S_l}
\def\rvector{Y}
\def\wvector{W}
\def\cvector{C}
\def\cword{C_{1},\dots,C_{l+n}}
\def\pcword{C_{l+1},\dots,C_{l+n}}
\def\randvector{R}

\def\compmat{\Phi}

%%%%%%%%%%%%%%%%%%%%%%%%%%%%%%%%%%%%%%%%%%%%%%%%%%%%%%%%%%%%%%%%%%%%%%%%%%%%%%%%%%%%%%%%%%%%%%%%%%%
%%%%%%%%%%%%%%%%%%%%%%%%%%%%%%%%%%%%%%%%%%%%%%%%%%%%%%%%%%%%%%%%%%%%%%%%%%%%%%%%%%%%%%%%%%%%%%%%%%%
%%%%%%%%%%%%%%%%%%%%%%%%%%%%%%%%%%%%%%%%%%%%%%%%%%%%%%%%%%%%%%%%%%%%%%%%%%%%%%%%%%%%%%%%%%%%%%%%%%%
%%%
%%%  日本語フォントをゴシックに、数式フォントを太字に変更する
%%%
\renewcommand{\kanjifamilydefault}{\gtdefault}
\renewcommand{\familydefault}{\sfdefault}

\setbeamerfont{title}{size=\large,series=\bfseries}
\setbeamerfont{frametitle}{size=\large,series=\bfseries}
%\setbeamertemplate{frametitle}[default][center]
\usefonttheme{professionalfonts} 

%\mathversion{bold} %数式フォントを太字に

%\def\vec#1{\mbox{\boldmath $#1$}}


%\logo{\includegraphics[width=2cm]{titech_logo.eps}}

%\setbeamertemplate{caption}[numbered]
%%%
%%% 著者など
%%%
\title[E2E Security with JS]{JavaScriptによるEnd-to-Endセキュリティ}
\subtitle{AESはどうやって使えばいいのか? (共通鍵暗号の使い方) 編}
\author[Jun Kurihara]{栗原 淳}
\institute[]{}
\date[\today]{\today}

%%%%%%%%%%%%%%%%%%%%%%%%%%%%%%%%%%%%%%%%%%%%%%%%%%%%%%%%%%%%%%%%%%%%%%%%%%%%%%%%%%%%%%%%%%%%%%%%%%%
%%%%%%%%%%%%%%%%%%%%%%%%%%%%%%%%%%%%%%%%%%%%%%%%%%%%%%%%%%%%%%%%%%%%%%%%%%%%%%%%%%%%%%%%%%%%%%%%%%%
%%%%%%%%%%%%%%%%%%%%%%%%%%%%%%%%%%%%%%%%%%%%%%%%%%%%%%%%%%%%%%%%%%%%%%%%%%%%%%%%%%%%%%%%%%%%%%%%%%%
%%%%%%%%%%%%%%%%%%%%%%%%%%%%%%%%%%%%%%%%%%%%%%%%%%%%%%%%%%%%%%%%%%%%%%%%%%%%%%%%%%%%%%%%%%%%%%%%%%%
%%%%%%%%%%%%%%%%%%%%%%%%%%%%%%%%%%%%%%%%%%%%%%%%%%%%%%%%%%%%%%%%%%%%%%%%%%%%%%%%%%%%%%%%%%%%%%%%%%%

\begin{document}

\begin{frame}
\titlepage
\end{frame}

%%%%%%%%%%%%%%%%%%%%%%%%%%%%%%%%%%%%%%%%%%%%%%%%%%%%%%%%%%%%%%%%%%%%%%%%%%%%%%%%%%%%%%%%%%%%%%%%%%%
\section{はじめに}
\begin{frame}
 \centering
 {\Large はじめに}
\end{frame}

\begin{frame}
\frametitle{はじめに}
前回 (第1回) は
\begin{itemize}
 \item End-to-End (E2E) セキュリティの原則と必要性
 \item WebサイトでのE2Eセキュリティ実践のため、JavaScriptでの暗号(AES)の利用のさわり
\end{itemize}
を勉強した。

\vspace{2ex}

E2Eセキュリティの重要性はわかった。

AESを使ってみることもできた。

\vspace{2ex}

でも、実際のAppで\alert{正しく・安全にAESを使うにはどうすべきなのか?}
\end{frame}

\begin{frame}
今回は\underline{正しく・安全に}AESを使ってみる方法、についてのお話。

\begin{block}{\small この講義で最終的に学びたいこと}
\begin{itemize}
\item パスワードを使ってAES暗号化はどうすればいいか?\footnote[frame]{\scriptsize RFC8018 PBES2 \url{https://tools.ietf.org/html/rfc8018}によるAES暗号化}
\item 固定バイナリ値を使ってAES暗号化はどうすればいいか?\footnote[frame]{\scriptsize RFC5869 HKDF \url{https://tools.ietf.org/html/rfc5869}による鍵導出とAES暗号化}
\end{itemize}
\end{block}

\vspace{2ex}

たったこれだけ。

\end{frame}

\begin{frame}

たったこれだけでも、気をつけなければならない\alert{「重要なお作法」}がある。

\vspace{2ex}

\alert{お作法を守る・守らないで安全性は大違い}なので、注意しなければならない。\footnote[frame]{世の中のソフトウェア、全くお作法を守ってないのがあって…危険…最近だとphpの\texttt{hash\_hkdf()}がお作法守ってなかった (2018年)。}
\end{frame}

\begin{frame}
\frametitle{この講義の対象と事前準備}
対象:
\begin{itemize}
\item 暗号・セキュリティ技術に興味がある初学者
\item Webに暗号技術を導入したいWeb系のエンジニア
\end{itemize}

\vspace{2ex}

必須ではないが触って楽しむのには必要な事前準備:
\begin{itemize}
\item Gitが使えるようになっていること
\item Node.jsが使えるようになっていること
\item Google Chrome系ブラウザ and/or Firefoxが利用可能なこと
\end{itemize}
\end{frame}


%%%%%%%%%%%%%%%%%%%%%%%%%%%%%%%%%%%%%%%%%%%%%%%%%%%%%%%%%%%%%%%%%%%%%%%%%%%%%%%%%%%%%%%%%%%%%%%%%%%
\section{AESの使い方 事始め}
\begin{frame}
\centering
{\Large AESの使い方 事始め}
\end{frame}

\begin{frame}
AESを使う際に気をつけるお作法は、ざっと3点。
\begin{enumerate}
 \item AESで使う鍵の\alert{ランダム具合}
 \item AESで使う鍵を\alert{総当りする際の大変さ}\footnote[frame]{1点目と2点目は似ているようで異なる。}
 \item AESの\alert{利用モードの安全性}
\end{enumerate}

\vspace{2ex}
つまりどういうこと?

\end{frame}

\begin{frame}
\frametitle{1: AESで使う鍵のランダム具合?}
$\Rightarrow$ \alert{過去の利用履歴も含めたランダムさ}のこと

\vspace{1ex}

\begin{block}{}
つまり…
\begin{itemize}
\item 過去に暗号化に使った鍵は\alert{二度と使わない}
\item 暗号化の鍵は、過去の鍵から\footnote[frame]{および未来に使う鍵からも}は容易に導出できないものへと\alert{毎回ランダム変更}する
\end{itemize}
ということ。
\end{block}

\end{frame}

\begin{frame}
…なぜか?

$\Rightarrow$鍵が1つ漏れてしまうと、過去の暗号化データまで一網打尽…。

\begin{center}
\includegraphics[width=0.9\linewidth]{Figs/pfs_bad_case.pdf}
\end{center}

\begin{block}{}
なので、万一鍵が1つ漏れちゃったとしても、他の暗号化データにまで影響が出ないことを保証しなきゃならない。\footnote[frame]{これを保証することを (Perfect) Forward Secrecyとか呼ぶ。}
\end{block}

\end{frame}

\begin{frame}
だが、\underline{暗号化毎のパスワード等のランダム変更は非現実的}。

\vspace{2ex}

$\Rightarrow$ \alert{固定パスワード等からランダムに鍵を導出する方法}を使う\footnote[frame]{PBKDF2 (RFC8018), HKDF (RFC5869)}。

\begin{center}
\includegraphics[width=0.9\linewidth]{Figs/pfs_kdf_case.pdf}
\end{center}

\end{frame}

\begin{frame}
\frametitle{2: AESで使う鍵を総当りする際の大変さ?}
$\Rightarrow$ 総当たり攻撃やるには膨大な計算が必要になるようにすること
\end{frame}

\begin{frame}
\frametitle{3: AESの利用モードの安全性?}

$\Rightarrow$ AESのAPIで設定できる利用モード('AES256-CBC'とか)と、そのパラメタ設定の適切な設定が必要。

\vspace{2ex}

\begin{block}{\small AESの「利用モード」}
AESの処理1回で暗号化できるのはたった16bytesにすぎない。

長いデータを連続で暗号化するために、\alert{暗号化処理を連続して組み合わせる方法}が利用モード。
\end{block}


\end{frame}

%%%%%%%%%%%%%%%%%%%%%%%%%%%%%%%%%%%%%%%%%%%%%%%%%%%%%%%%%%%%%%%%%%%%%%%%%%%%%%%%%%%%%%%%%%%%%%%%%%%
\section{AESの使い方: とりあえず暗号化してみよう}
\begin{frame}
\centering
{\Large AESの使い方: とりあえず暗号化してみよう}
\end{frame}


\begin{frame}
\frametitle{パスワードで暗号化してみる}
pbkdf使え
\end{frame}

\begin{frame}
\frametitle{バイナリ鍵で暗号化してみる}
hkdf使え
\end{frame}

\begin{frame}
\frametitle{危ない暗号化モードで暗号化してみる}
ecbとか論外だからctrとか使え
\end{frame}

%%%%%%%%%%%%%%%%%%%%%%%%%%%%%%%%%%%%%%%%%%%%%%%%%%%%%%%%%%%%%%%%%%%%%%%%%%%%%%%%%%%%%%%%%%%%%%%%%%%
\section{AESの使い方: 細かめの解説}
\begin{frame}
\centering
{\Large AESの使い方: 細かめの解説}
\end{frame}

\begin{frame}
\frametitle{PBKDF2の使い方 in JavaScript}
jscuなら動くよ
\end{frame}

\begin{frame}
\frametitle{HKDFの使い方 in JavaScript}

\end{frame}

\begin{frame}
\frametitle{暗号化モードの設定}
ctr云々。

jsでのサポート具合を列挙
\end{frame}



%%%%%%%%%%%%%%%%%%%%%%%%%%%%%%%%%%%%%%%%%%%%%%%%%%%%%%%%%%%%%%%%%%%%%%%%%%%%%%%%%%%%%%%%%%%%%%%%%%%
\section{まとめ}
\begin{frame}
 \centering
 {\Large まとめ}
\end{frame}

\begin{frame}
\frametitle{まとめ}
お疲れ様でした。

\begin{itemize}
\item 
\end{itemize}

\vspace{2ex}

次回以降…リクエスト次第ですが、
\begin{itemize}
\item 公開鍵暗号とその使い方
\item 「情報が改ざんされてない」ことを保証するために(電子署名とMAC)
\item RFCとアルゴリズム・フォーマット
\end{itemize}
などを予定。
\end{frame}


\begin{frame}
\frametitle{宣伝 1}
E2E暗号化ファイル転送サービス「iTransfy」を提供しています。
\end{frame}


\begin{frame}
\frametitle{宣伝 2}
Zettantではイケイケの仲間を募集しています。
\end{frame}



%%%%%%%%%%%%%%%%%%%%%%%%%%%%%%%%%%%%%%%%%%%%%%%%%%%%%%%%%%%%%%%%%%%%%%%%%%%%%%%%%%%%%%%%%%%%%%%%%%%

\backupbegin
\section{Backup}

\begin{frame}
 
\end{frame}

\begin{frame}
 \begin{enumerate}
  \item 今回は共通鍵暗号
  \item 公開鍵暗号\& Hybrid Encryption
  \item ハッシュ・署名とHMAC
 \item 超マニアック講座:RFCとアルゴリズム・フォーマット
 \end{enumerate}
\end{frame}

\begin{frame}
\frametitle{Appendix}
This page is not counted.
\end{frame}
\backupend
\end{document}
%%%%%%%%%%%%%%%%%%%%%%%%%%%%%%%%%%%%%%%%%%%%%%%%%%%%%%%%%%%%%%%%%%%%%%%%%%%%%%%%%%%%%%%%%%%%%%%%%%%
%%%%%%%%%%%%%%%%%%%%%%%%%%%%%%%%%%%%%%%%%%%%%%%%%%%%%%%%%%%%%%%%%%%%%%%%%%%%%%%%%%%%%%%%%%%%%%%%%%%
%%%%%%%%%%%%%%%%%%%%%%%%%%%%%%%%%%%%%%%%%%%%%%%%%%%%%%%%%%%%%%%%%%%%%%%%%%%%%%%%%%%%%%%%%%%%%%%%%%%
%%%%%%%%%%%%%%%%%%%%%%%%%%%%%%%%%%%%%%%%%%%%%%%%%%%%%%%%%%%%%%%%%%%%%%%%%%%%%%%%%%%%%%%%%%%%%%%%%%%
%%%%%%%%%%%%%%%%%%%%%%%%%%%%%%%%%%%%%%%%%%%%%%%%%%%%%%%%%%%%%%%%%%%%%%%%%%%%%%%%%%%%%%%%%%%%%%%%%%%
%%%%%%%%%%%%%%%%%%%%%%%%%%%%%%%%%%%%%%%%%%%%%%%%%%%%%%%%%%%%%%%%%%%%%%%%%%%%%%%%%%%%%%%%%%%%%%%%%%%
