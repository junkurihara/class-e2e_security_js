%%%%%%%%%%%%%%%%%%%%%%%%%%%%%%%%%%%%%%%%%%%%%%%%%%%%%%%%%%%%%%%%%%%%%%%%%%%%%%%%%%%%%%%%%%%%%%%%%%%
%%%%%%%%%%%%%%%%%%%%%%%%%%%%%%%%%%%%%%%%%%%%%%%%%%%%%%%%%%%%%%%%%%%%%%%%%%%%%%%%%%%%%%%%%%%%%%%%%%%
%%%%%%%%%%%%%%%%%%%%%%%%%%%%%%%%%%%%%%%%%%%%%%%%%%%%%%%%%%%%%%%%%%%%%%%%%%%%%%%%%%%%%%%%%%%%%%%%%%%
\documentclass[12pt,dvipdfmx]{beamer}
%%%%%%%%%%%%%%%%%%%%%%%%%%%%%%%%%%%%%%%%%%%%%%%%%%%%%%%%%%%%%%%%%%%%%%%%%%%%%%%%%%%%%%%%%%%%%%%%%%%%%%
% pdfの栞・プロパティの字化けを防ぐ
\usepackage{atbegshi}
%\AtBeginShipoutFirst{\special{pdf:tounicode 90ms-RKSJ-UCS2}} %Windows
\AtBeginShipoutFirst{\special{pdf:tounicode EUC-UCS2}} %Linux, Mac
\usepackage{hyperref}
%%%%%%%%%%%%%%%%%%%%%%%%%%%%%%%%%%%%%%%%%%%%%%%%%%%%%%%%%%%%%%%%%%%%%%%%%%%%%%%%%%%%%%%%%%%%%%%%%%%%%%
%%%
%%% テーマの指定、省略時は default になる
%%%

 % フレームの指定、省略可
%%%%%%%%%%%%%%%%%%%%%%%%%%%% THEME
  %\usetheme{AnnArbor}
  %\usetheme{Antibes}
  %\usetheme{Bergen}
  %\usetheme{Berkeley}
  %\usetheme{Berlin}
  \usetheme{Boadilla}
  %\usetheme{boxes}
  %\usetheme{CambridgeUS}
  %\usetheme{Copenhagen}
  %\usetheme{Darmstadt}
  %\usetheme{default}
  %\usetheme{Dresden}
  %\usetheme{Frankfurt}
  %\usetheme{Goettingen}
  %\usetheme{Hannover}
  %\usetheme{Ilmenau}
  %\usetheme{JuanLesPins}
  %\usetheme{Luebeck}
  %\usetheme{Madrid}
  %\usetheme{Malmoe}
  %\usetheme{Marburg}
  %\usetheme{Montpellier}
  %\usetheme{PaloAlto}
  %\usetheme{Pittsburgh}
  %\usetheme{Rochester}
  %\usetheme{Singapore}
  %\usetheme{Szeged}
  %\usetheme{Warsaw}

% 省略可
%%%%%%%%%%%%%%%%%%%%%%%%%%%% COLOR THEME
  %\usecolortheme{albatross}
  %\usecolortheme{beetle}
  %\usecolortheme{crane}
  %\usecolortheme{default}
  %\usecolortheme{dolphin}
  %\usecolortheme{dove}
  %\usecolortheme{fly}
  %\usecolortheme{lily}
  %\usecolortheme{orchid}
  %\usecolortheme{rose}
  %\usecolortheme{seagull}
  %\usecolortheme{seahorse}
  %\usecolortheme{sidebartab}
  %\usecolortheme{structure}
  %\usecolortheme{whale}

% ヘッダ、フッタ、フレーム等を指定、省略可
  %%%%%%%%%%%%%%%%%%%%%%%%%%%% OUTER THEME
  %\useoutertheme{default}
  %\useoutertheme{infolines}
  %\useoutertheme{miniframes}
  %\useoutertheme{shadow}
  %\useoutertheme{sidebar}
  %\useoutertheme{smoothbars}
  %\useoutertheme{smoothtree}
  %\useoutertheme{split}
  %\useoutertheme{tree}

% タイトル、section, itemize/enumerate 環境、
% theorem 環境、図, 参考文献などのスタイルを指定、
% 省略可
  %%%%%%%%%%%%%%%%%%%%%%%%%%%% INNER THEME
  %\useinnertheme{circles}
  %\useinnertheme{default}
  %\useinnertheme{inmargin}
  \useinnertheme{rectangles}
  %\useinnertheme{rounded}


%\usefonttheme{}	% 省略可
%\logo{}		% 省略可

%%%%%%%%%%%%%%%%%%%%%%%%%%%%%%%%%%%%%%%%%%%%%%%%%%%%%%%%%%%%%%%%%%%%%%%%%%%%%%%%%%%%%%%%%%%%%%%%%%%
%%%%%%%%%%%%%%%%%%%%%%%%%%%%%%%%%%%%%%%%%%%%%%%%%%%%%%%%%%%%%%%%%%%%%%%%%%%%%%%%%%%%%%%%%%%%%%%%%%%
%%%%%%%%%%%%%%%%%%%%%%%%%%%%%%%%%%%%%%%%%%%%%%%%%%%%%%%%%%%%%%%%%%%%%%%%%%%%%%%%%%%%%%%%%%%%%%%%%%%
% navi. symbolsは目立たないが,dvipdfmxを使うと機能しないので非表示に
\setbeamertemplate{navigation symbols}{}

% 各種パッケージ
\usepackage{graphicx}
%\usepackage{url,cite}
\usepackage{amsmath}
\usepackage{amsthm} \theoremstyle{definition} %theorem環境が斜体になるので注意
\usepackage{amssymb} % AMS-TeX
\usepackage{setspace}

% \AtBeginSection[] % Do nothing for \section*
% { \begin{frame}<beamer> \frametitle{}
%    \tableofcontents[currentsection,subsectionstyle=hide]
%  \end{frame} } 

%appendixをページカウントしない
\newcommand{\backupbegin}{
   \newcounter{framenumberappendix}
   \setcounter{framenumberappendix}{\value{framenumber}}
}
\newcommand{\backupend}{
   \addtocounter{framenumberappendix}{-\value{framenumber}}
   \addtocounter{framenumber}{\value{framenumberappendix}} 
}

%%%%%%%%%%%%%%%%%%%%%%%%%%%%%%%%%%%%%%%%%%%%%%%%%%%%%%%%%%%%%%%%%%%%%%%%%%%%%%%%%%%%%%%%%%%%%%%%%%%%%%
% 本文・数式フォント
%\usepackage{palatino,mathpazo}
%\usepackage{times,mathptmx}
\usepackage[varg]{txfonts}
%\usepackage[varg]{pxfonts}

% \mathcal(\cal)の扱い
%\DeclareMathAlphabet{\mathcal}{OMS}{cmsy}{m}{n} %computer modern
%\DeclareMathAlphabet{\mathcal}{OMS}{txsy}{m}{n} %txfont
%\usepackage[psamsfonts]{eucal} % euler

% mathptmx時に数式モードのvをtxfontから借りる
% \DeclareSymbolFont{lettersA}{U}{txmia}{m}{it}
% \SetSymbolFont{lettersA}{bold}{U}{txmia}{bx}{it}
% \DeclareFontSubstitution{U}{txmia}{m}{it}
% \DeclareMathSymbol{v}{\mathalpha}{lettersA}{"33} %"

\usepackage{multirow}

%上線 widebar, Widebar
\usepackage{accents}
\makeatletter
\def\widebar{\accentset{{\cc@style\underline{\mskip11mu}}}}
\makeatother


%%%%%%%%%%%%%%%%%%%%%%%%%%%%%%%%%%%%%%%%%%%%%%%%%%%%%%%%%%%%%%%%%%%%%%%%%%%%%%%%%%%%%%%%%%%%%%%%%%%%%%

% 定理環境
% \newtheorem{theorem}{Theorem}
% \newtheorem{lemma}[theorem]{Lemma}
% \newtheorem{corollary}[theorem]{Corollary}
% \newtheorem{definition}[theorem]{Definition}
% \newtheorem{example}[theorem]{Example}
\newtheorem{proposition}{Proposition}
\newtheorem{remark}{Remark}

%%%%%%%%%%%%%%%%%%%%%%%%%%%%%%%%%%%%%%%%%%%%%%%%%%%%%%%%%%%%%%%%%%%%%%%%%%%%%%%%%%%%%%%%%%%%%%%%%%%%%%
% 各種コマンド定義等
\def\Fig#1{Fig.\@\ref{#1}}
\def\Table#1{Table~\ref{#1}}
\def\Eq#1{Eq.\@(\ref{#1})}
\def\Eqs#1{Eqs.\@(\ref{#1})}
\def\Thm#1{Theorem~\ref{#1}}
\def\Lma#1{Lemma~\ref{#1}}
\def\Sect#1{Section~\ref{#1}}
\def\Rmk#1{Remark~\ref{#1}}
\def\Prop#1{Proposition~\ref{#1}}
\def\Coro#1{Corollary~\ref{#1}}
\def\Def#1{Definition~\@\ref{#1}}
\def\Prob#1{Problem~\@\ref{#1}}
\def\ie{{i.\@e.\@,~}}
\def\eg{{e.\@g.\@,~}}
\def\etal{{et al.}}

% 数式環境用
\def\rank{\mathsf{rank}\, }
\def\dim{\mathsf{dim}\, }
\def\rspace{\mathsf{span}}
\def\supp{\mathsf{supp}}
%\def\vec#1{\mathbf{#1}}
\def\F{\mathbb{F}}
\def\wt{\mathsf{wt}}
\def\c{\mathcal{C}}
\def\dc{\mathcal{C}^{\perp}}
\def\d{\mathcal{D}}
\def\dd{\mathcal{D}^{\perp}}
\def\g{\mathcal{G}}
\def\dg{\mathcal{G}^{\perp}}
\def\p{\mathcal{P}}
% \def\rspace{\mathsf{span}}
\def\supp{\mathsf{supp}}
\def\ker{\mathsf{Ker\ }}

%\def\bari#1{\{\widebar{#1}\}}
\def\bari#1{\,\overline{{\!\{#1\}\!}}\,}
%\def\bari#1{\bar{\{#1\}}}
\def\vecxi{Z_{\bari{i}}}
%\def\vecsxi{\vec{z}_i}
\def\tvector{X}
\def\tpackets{X_1,\dots,X_n}
\def\mvector{S}
\def\mpackets{S_1,\dots,S_l}
\def\rvector{Y}
\def\wvector{W}
\def\cvector{C}
\def\cword{C_{1},\dots,C_{l+n}}
\def\pcword{C_{l+1},\dots,C_{l+n}}
\def\randvector{R}

\def\compmat{\Phi}

%%%%%%%%%%%%%%%%%%%%%%%%%%%%%%%%%%%%%%%%%%%%%%%%%%%%%%%%%%%%%%%%%%%%%%%%%%%%%%%%%%%%%%%%%%%%%%%%%%%
%%%%%%%%%%%%%%%%%%%%%%%%%%%%%%%%%%%%%%%%%%%%%%%%%%%%%%%%%%%%%%%%%%%%%%%%%%%%%%%%%%%%%%%%%%%%%%%%%%%
%%%%%%%%%%%%%%%%%%%%%%%%%%%%%%%%%%%%%%%%%%%%%%%%%%%%%%%%%%%%%%%%%%%%%%%%%%%%%%%%%%%%%%%%%%%%%%%%%%%
%%%
%%%  日本語フォントをゴシックに、数式フォントを太字に変更する
%%%
\renewcommand{\kanjifamilydefault}{\gtdefault}
\renewcommand{\familydefault}{\sfdefault}

\setbeamerfont{title}{size=\large,series=\bfseries}
\setbeamerfont{frametitle}{size=\large,series=\bfseries}
%\setbeamertemplate{frametitle}[default][center]
\usefonttheme{professionalfonts} 

%\mathversion{bold} %数式フォントを太字に

%\def\vec#1{\mbox{\boldmath $#1$}}


%\logo{\includegraphics[width=2cm]{titech_logo.eps}}

%\setbeamertemplate{caption}[numbered]
%%%
%%% 著者など
%%%
\title[E2E Security with JS Appendix]{JavaScriptによるEnd-to-Endセキュリティ}
\subtitle{標準規格とセキュリティエンジニアリング}
\author[Jun Kurihara]{栗原 淳}
\institute[]{}
\date[\today]{\today}

%%%%%%%%%%%%%%%%%%%%%%%%%%%%%%%%%%%%%%%%%%%%%%%%%%%%%%%%%%%%%%%%%%%%%%%%%%%%%%%%%%%%%%%%%%%%%%%%%%%
%%%%%%%%%%%%%%%%%%%%%%%%%%%%%%%%%%%%%%%%%%%%%%%%%%%%%%%%%%%%%%%%%%%%%%%%%%%%%%%%%%%%%%%%%%%%%%%%%%%
%%%%%%%%%%%%%%%%%%%%%%%%%%%%%%%%%%%%%%%%%%%%%%%%%%%%%%%%%%%%%%%%%%%%%%%%%%%%%%%%%%%%%%%%%%%%%%%%%%%
%%%%%%%%%%%%%%%%%%%%%%%%%%%%%%%%%%%%%%%%%%%%%%%%%%%%%%%%%%%%%%%%%%%%%%%%%%%%%%%%%%%%%%%%%%%%%%%%%%%
%%%%%%%%%%%%%%%%%%%%%%%%%%%%%%%%%%%%%%%%%%%%%%%%%%%%%%%%%%%%%%%%%%%%%%%%%%%%%%%%%%%%%%%%%%%%%%%%%%%

\begin{document}

\begin{frame}
\titlepage
\end{frame}

%%%%%%%%%%%%%%%%%%%%%%%%%%%%%%%%%%%%%%%%%%%%%%%%%%%%%%%%%%%%%%%%%%%%%%%%%%%%%%%%%%%%%%%%%%%%%%%%%%%
\section{はじめに}
\begin{frame}
 \centering
 {\Large はじめに}
\end{frame}

\begin{frame}
\frametitle{はじめに}
この資料は、「JavaScriptによるEnd-to-Endセキュリティ」の補足資料である。

「標準規格」そのもの、および一連の勉強会にて話題に上げた鍵フォーマットや標準アルゴリズムについてを解説を与える。
\end{frame}

%%%%%%%%%%%%%%%%%%%%%%%%%%%%%%%%%%%%%%%%%%%%%%%%%%%%%%%%%%%%%%%%%%%%%%%%%%%%%%%%%%%%%%%%%%%%%%%%%%%
\section{セキュリティエンジニアリングと標準規格}
\begin{frame}
\centering
 {\Large セキュリティ関連の標準規格}
\end{frame}

\begin{frame}
\frametitle{はじめに: セキュリティエンジニアリングと標準規格}
\begin{block}{セキュリティエンジニアリング\footnote[frame]{\scriptsize \url{https://www.ipa.go.jp/security/awareness/vendor/software.html}} }
ソフトウェア・アプリケーション開発において、セキュリティを考慮したエンジニアリング、あるいはセキュリティエンジニアリングを行うためには、要件定義・設計・実装・テストの段階ごとに種々のセキュリティ関連事項を検討する必要がある。
\end{block}
\begin{center}
\includegraphics[width=0.8\linewidth]{FigsAppendix/security-eng-2.png}
\end{center}
\end{frame}

\begin{frame}
一方で、標準規格は:
\begin{block}{標準規格となるアルゴリズム・プロトコル}
国家や団体の標準文書に載せるために、\alert{安全性・相互接続性・効率性を分析し、安全性や将来性などがある程度確保}されたもの、と言える。
\end{block}

\begin{center}
 $\Downarrow$
\end{center}

すなわち、\underline{標準規格を要件に応じて適切に選択}し、設計・実装を行うことで脆弱性低減、使いやすさとの両立や相互運用可能性の担保が容易になると言える。

\vspace{1ex}

つまり、最新の標準規格とその推奨される利用方法\footnote[frame]{\scriptsize なぜそのように利用するのか・載っているのか、も正しく知っておく必要がある。悪い例は、脆弱性はあるが互換性のために残っているRFC8017のRSAES-PKCS1-v1\_5。}を把握しておくことで、\alert{効率的なセキュリティエンジニアリングが行える}。
\end{frame}

\begin{frame}
 次ページからは、実際のセキュリティ関連標準について紹介する。
\end{frame}


\begin{frame}
\frametitle{PKCS (Public Key Cryptography Standards)}
\begin{block}{PKCSとは?}
RSA Security社\footnote[frame]{\scriptsize RSA暗号を作ったRivest-Shamir-Adlemanの会社。}の研究部門RSA Labsが策定・公開している、公開鍵暗号を中心とする一連のセキュリティ技術標準のこと。比較的古い、\underline{枯れた標準}になる。
\end{block}
\begin{itemize}
 \item \#1,...\#15の15本が存在\footnote[frame]{\scriptsize 破棄・廃盤も含む。}。
 \item 暗号化・署名生成の手続き・アルゴリズム、データフォーマットの規定など、いわゆる「ローレベル」の技術標準が中心。
 \item RSA暗号標準を定める\#1を代表に、重要なものはメンテされ続けている。が、他の新標準規格で代替されるものなどは破棄・管理移譲されている。
\end{itemize}
\end{frame}

\begin{frame}
PKCSは、その名目上「私企業」が策定した技術標準。

しかし、採用された技術は、十分にセキュリティ評価されていると見做されるものが多く、また他の技術標準に採用・移植・継承されている。
特に多くはIETF RFCの管理下へ継承されているようだ。
\end{frame}

\begin{frame}
PKCS文書一覧(1/2)
\begin{table}
\scriptsize
\begin{tabular}{|l|l|p{21ex}|p{40ex}|}
\hline
 & Ver. & 名称 & 内容 \\
\hline\hline
PKCS \#1 & v2.2 & RSA Cryptography Specifications\footnote[frame]{\scriptsize \url{https://tools.ietf.org/html/rfc8017}} & RSA鍵ペアの構造、暗号化手法、署名手法を策定。\\
\hline
PKCS \#3 & v1.4 & Diffie–Hellman Key Agreement Standard & Diffie-Hellman鍵交換の仕様を策定。RFCではInternet Key Exchange (IKE)へ継承(?)。\\
\hline
PKCS \#5 & v2.1 & Password-Based Cryptography Specification\footnote[frame]{\scriptsize \url{https://tools.ietf.org/html/rfc8018}} & パスワードからの鍵導出手法、暗号化手法 (PBKDF1/2, PBES1/2) の策定。\\
\hline
PKCS \#6 & 廃止 & Extended-Certificate Syntax Standard & X.509v1証明書の拡張。X.509v3へ統合されて廃止。\\
\hline
PKCS \#7 & 廃止(?) & Cryptographic Message Syntax Standard\footnote[frame]{\scriptsize \url{https://tools.ietf.org/html/rfc2315}} & 暗号メッセージ構文を策定。S/MIMEに利用。より新しい仕様 (RFC5652) により廃止(?)。\\
\hline
PKCS \#8 & 廃止(?) & Private-Key Information Syntax Specification\footnote[frame]{\scriptsize \url{https://tools.ietf.org/html/rfc5208}} & 秘密鍵フォーマットを策定。より新しい仕様 (RFC5968) によりv1.2で廃止(?)。\\
\hline
PKCS \#9 & v2.0 & Selected Object Classes and Attribute Types\footnote[frame]{\scriptsize \url{https://tools.ietf.org/html/rfc2985}} & 各種フォーマットにおける「属性」タイプを策定。\\
\hline
\end{tabular}
\end{table}
\end{frame}

\begin{frame}
PKCS文書一覧(2/2)
\begin{table}
\scriptsize
\begin{tabular}{|l|l|p{21ex}|p{40ex}|}
\hline
 & Ver. & 名称 & 内容 \\
\hline\hline
PKCS \#10 & v1.7 & Certification Request Syntax Specification\footnote[frame]{\scriptsize \url{https://tools.ietf.org/html/rfc2986} + \url{https://tools.ietf.org/html/rfc5967}} & 証明書リクエスト構文を策定。元はPKCSのみで策定されていたが、利用されるメディアタイプをRFC5967で拡張。\\
\hline
PKCS \#11 & v2.40 & Cryptographic Token Interface & Cryptokiとしても知られる、暗号トークン(H/Wセキュリティモジュール)インターフェースの仕様を策定。OASIS PKCS 11 Technical Committeeへ継承。\\
\hline
PKCS \#12 & v1.1 & Personal Information Exchange Syntax Standard\footnote[frame]{\scriptsize \url{https://tools.ietf.org/html/rfc7292}} & パスワード暗号化された秘密鍵、公開鍵証明書の構文を策定。IETF IESG管理下へ継承。\\
\hline
PKCS \#15 & v1.1 & Cryptographic Token Information Format Standard & 暗号トークン向け、ユーザ特定標準仕様の策定。ICカード部分はISO/IEC 7816-15へ移譲。\\
\hline
\end{tabular}
\end{table}
策定中のまま立ち消えたものなどは削除。

IETF RFCなどへRepublication、あるいは継承されて新しい標準になっている。
\end{frame}

\begin{frame}
\frametitle{NIST FIPS および SP800\footnote[frame]{参考: https://www.ipa.go.jp/security/publications/nist/}}
\begin{block}{NIST FIPS/SP800とは?}
米国国立標準技術研究所 (NIST; National Institute of Standards and Technology) の発行する文書のこと。
\begin{itemize}
 \item \alert{FIPS; Federal Information Processing Standards}: 米国商務長官の承認の下、NISTが公布した情報セキュリティ関連の米国の標準規格文書。詳細な基準や要求事項、ガイドラインが記載されている。
 \item SP800; Special Publication: 米国政府がセキュリティ対策を実施する際に参考とすることを前提とした、コンピュータセキュリティ関係のレポート。
\end{itemize}
\end{block}
\end{frame}

\begin{frame}
すなわちNIST FIPSは、米国ローカルの標準規格と言える。

\begin{itemize}
\item 多くは、他の国別標準規格同様にANSI/ISO/IEEE等で広く使われていた既存規格を引き継ぐ。
\item 一部はNIST FIPS独自に公募・評価・策定した独自規格。代表的なものは、公募されてきた`Rijndael'という新暗号アルゴリズムを採用したFIPS 197; Advanced Encryption Standard (AES)。
\end{itemize}
\end{frame}

\begin{frame}
\frametitle{IETF RFC}
\begin{block}{RFC (Request for Comments) とは?}
「インターネット技術」全般の国際標準を議論策定するグループ IETF (Internet Engineering Task Force) で議論策定された、「インターネット技術標準」および「その他」\footnote[frame]{\scriptsize \url{https://www.nic.ad.jp/ja/rfc-jp/RFC-Category.html}}の広範な内容を扱う文書(群)のこと。
\end{block}

詳細仕様を策定するITU-TやISOと異なり、「まずは動作させる」ことを目的として実験的な「Rough」な仕様をまず策定することが特徴。
\end{frame}

\begin{frame}
RFCは5つのカテゴリに分類される:
\begin{itemize}
 \item \structure{Standards Track}: Proposed Standard $\rightarrow$ Internet Standard という策定過程を経る「インターネット標準技術仕様」の文書。
 \item \structure{Informational}: すでにデファクト標準であったり、インターネット標準の議論・策定において有益として公開されたもの。例えば、RSAセキュリティ社のPKCS\#1 v2.1 = RFC8017。
 \item \structure{Experimental}: デファクト標準を狙うような、研究等の目的で公開される技術仕様文書。
 \item \structure{Historical}: 過去の記録として残す情報としての文書。
 \item \structure{Best Current Practice}: 現状のベストプラクティスをまとめた仕様文書。
\end{itemize}

特にStandard Track, Informational, Experimentalに関して、PKCS等の他標準を引き継いだり、新たな技術標準を定めた文書が策定される。
\end{frame}

\begin{frame}
セキュリティ関係のRFC化の事例:
\begin{itemize}
 \item 事例1: OpenID Connectによって利用される鍵や署名、暗号化の仕様: JWS\footnote[frame]{\scriptsize JSON Web Signature \url{https://tools.ietf.org/html/rfc7515}}, JWE\footnote[frame]{\scriptsize JSON Web Encryption \url{https://tools.ietf.org/html/rfc7516}}, JWK\footnote[frame]{\scriptsize JSON Web Key \url{https://tools.ietf.org/html/rfc7517}}, JWT\footnote[frame]{\scriptsize JSON Web Token \url{https://tools.ietf.org/html/rfc7519}}について、OpenID Foundationのメンバにより、RFC Standards Trackとして国際標準化。
 \item 事例2: PKCS\#1, \#5, \#9等のRSAセキュリティ社の独自標準は、InformationalとしてRFC化。
 \item 事例3: HTTPSを支えるTLS v1.3\footnote[frame]{\scriptsize \url{https://tools.ietf.org/html/rfc8446}}は、Standards TrackとしてRFC化。
\end{itemize}

\end{frame}

\begin{frame}
\frametitle{ISO/IEC JTC 1}
\begin{block}{ISO\footnote[frame]{\scriptsize 国際標準化機構; International Organization for Standardization}/IEC\footnote[frame]{\scriptsize 国際電気標準会議; International Electrotechnical Commission} JTC (Joint Technical Committee) 1 とは}
情報技術の分野で国際標準化を行うための、ISOとIECの第一合同技術委員会のこと。セキュリティ技術は、第27 subcommittee (SC27)。
\end{block}

ISO/IEC JTC 1は各国が提案を持ち寄り議論が進められる国際標準化団体。他の標準規格からの引継ぎだけではなく、国際会議等で発表された新しいメカニズムやアルゴリズムも標準規格として提案される。
\end{frame}

\begin{frame}
ISO/IEC JTC1 SC27は5つのWorking Groupで構成される\footnote[frame]{\scriptsize \url{https://www.itscj.ipsj.or.jp/hyojunka/h_sn_member/h_sn_katsudo/h_sn_katsudo2013/sc27_2013.html}}。特にWG2で扱われるものが、セキュリティ「技術」や「メカニズム」の標準規格となる。

\begin{itemize}
 \item WG 1 情報セキュリティマネジメントシステム (いわゆるISMS)
 \item \alert{WG 2 暗号とセキュリティメカニズム}: 暗号アルゴリズム・プロトコルの標準規格。他の標準規格からの引継ぎだけでなく、各国から新たに提案されたものを議論し、標準規格を決定している。
 \item WG3 セキュリティの評価・試験・仕様
 \item WG4 セキュリティコントロールとサービス
 \item WG5 アイデンティティ管理とプライバシー技術
\end{itemize}

\end{frame}


\begin{frame}
\frametitle{W3C (World Wide Web Consortium)}
\begin{block}{W3Cとは?}
WWWで用いられるWeb技術の標準化、相互運用性の確保を目的とする団体。ブラウザのAPIやHTML, XML, DOM等の標準規格を「勧告 (Recommendation)」として策定する。
\end{block}

\end{frame}

\begin{frame}
W3Cのセキュリティ関連WG (Working Group) で有名な活動として、以下のような国際標準化策定が上げられる。
\begin{itemize}
 \item \structure{WebCrypto WG}\footnote[frame]{\scriptsize \url{https://www.w3.org/2012/webcrypto/} 現状はClose。}: WebCrypto APIを策定、勧告として国際標準化。
 \item \structure{WebAuthn WG}\footnote[frame]{\scriptsize \url{https://www.w3.org/blog/webauthn/}}: FIDOアライアンスの技術仕様を勧告として国際標準化 \footnote[frame]{\scriptsize 対象はブラウザ・端末・認証サーバの連携プロトコルであるFIDO2 WebAuthn \url{https://www.w3.org/2019/03/pressrelease-webauthn-rec.html.ja}。デバイス連携プロトコルであるFIDO2 CTAPはITU-Tで国際標準化。}
\end{itemize} 
\end{frame}

\begin{frame}
\frametitle{ITU-T SG17}
\begin{block}{\footnotesize ITU-T (International Telecommunication Union Telecommunication Standardization Sector) SG17 (Study Group 17) とは?\footnote[frame]{\scriptsize \url{https://www.ituaj.jp/wp-content/uploads/2016/07/2016_08-06-spotITU-T.pdf}}}
\begin{itemize}
 \item ITU-T: ITU (International Telecom. Union; 国際電気通信連合) における通信分野の標準技術を策定する「電気通信標準化部門」。策定された標準は「勧告」として発行される。
 \item Study Group 17: ITU-Tにおいてセキュリティ関連勧告作成の中心となるグループ。
\end{itemize}
\end{block}

\end{frame}

\begin{frame}
ITU-T SG17で取り扱う技術は、SDN・IoT・ITS・クラウドのセキュリティ技術、SPAM対策、ID管理技術、認証技術、テレバイオメトリクスなど、ソフトウェア実装のためのアルゴリズムというより\alert{「通信事業者」や「通信端末」を対象とした分野の技術}。

\vspace{1ex}

昨今だと、「FIDOアライアンスの技術仕様を勧告として国際標準化\footnote[frame]{\scriptsize 対象はデバイス連携プロトコルであるFIDO UAF 1.1およびCTAP \url{https://fidoalliance.org/fido-alliance-specifications-now-adopted-as-itu-international-standards/}。Web関連プロトコルであるFIDO2 WebAuthnはW3Cで国際標準化。}」している。

\end{frame}


\begin{frame}
\frametitle{その他; 各国の推奨技術リストとしての標準規格}
\small
\begin{itemize}
 \item CRYPTOREC\footnote[frame]{\scriptsize Cryptography Research and Evaluation Committee} \\
電子政府推奨暗号リストを作り、その実装や運用方法も含めて安全性を調査・評価・監視・検討するプロジェクト (2000年〜)
 \item NESSIE\footnote[frame]{\scriptsize New European Schemes for Signature, Integrity, and Encryption}\\
EUの制定した暗号標準リストを策定するプロジェクト (2000年〜)
\end{itemize}
基本的には、「評価検討した結果、\underline{既存}のアルゴリズム・プロトコルのどれそれを標準として採用する」という\alert{「推奨技術リスト」の策定プロジェクト}だと思って差し支えない。
\end{frame}

\begin{frame}
\small
 
\begin{block}{\small 「各国独自」という意味}
推奨技術リストへ採用されたアルゴリズム・プロトコルは、「その国において正しく評価された比較的安全なもの」というお墨付きを得る。


\underline{セキュリティ技術は国防上重要な意味を持つ}ため、このお墨付きは、その技術を自国で利用して良いものかどうかを判定するもの、と言える。
\end{block}

仕様の詳細はIETF (RFC), ISO, NIST公募など国際的に比較的オープンな場でまず評価・採用・策定される\footnote[frame]{\scriptsize 例外は存在する。元々PKCSはRSA Labs.の独自標準を公開したものだったが、IETFの公開の場でInternet Draftの形で標準化されてきている。}。

\vspace{1ex}

その後、各国が独自に調査検討して推奨技術リストとして採用する、というケースが多い。

\end{frame}

\begin{frame}
\frametitle{その他; 諸々}
\begin{itemize}
 \item FIDO Alliance: フォーラム標準を定める業界団体。国際標準ではない。
 \item OpenID Foundation: 上記同様。
 \item Ecma International: JavaScriptのメイン機能の標準規格 (ECMAScript) を策定する業界団体。JS以外にも多種の規格策定、およびEcma規格の国際標準化を行っている。\footnote[frame]{\scriptsize 例えば: Ecma規格ECMA-334 ``C\#'' は、国際規格ISO/IEC 23270として標準化。}
\end{itemize}
\end{frame}


\section{公開鍵・秘密鍵等の表現形式}
\begin{frame}
\centering
{\Large 公開鍵・秘密鍵等の表現形式}
\end{frame}

\begin{frame}
\frametitle{公開鍵・秘密鍵・証明書等の形式}
公開鍵・秘密鍵・証明書などを表すための代表的な表現形式:
\begin{itemize}
 \item DER (Distinguished Encoding Rules) 形式: ANS.1で記述されるエンコード方法。RSA暗号の公開鍵・秘密鍵、楕円曲線暗号の公開鍵・秘密鍵など、それぞれエンコード方法を規定。
 \item \structure{PEM (Privacy Enhanced Mail\footnote[frame]{\scriptsize 元々メールを暗号化したデータのエンコード方法だったので。})}: DERのBase64テキスト。
 \item \structure{SECG SEC1形式}: (楕円曲線暗号の鍵のみ) 業界団体SECGで定められた公開鍵・秘密鍵のバイナリ表現方法。\footnote[frame]{\scriptsize \url{http://www.secg.org/sec1-v2.pdf}}
 \item JWK/JWE/JWS (JSON Web Key/Encryption/Signature) 形式: JSON形式での表現方法。\footnote[frame]{\scriptsize \url{https://tools.ietf.org/html/rfc7517}, \url{https://tools.ietf.org/html/rfc7516}, \url{https://tools.ietf.org/html/rfc7515}}
\end{itemize}
基本、JavaScriptでは鍵をJWKとして使うことが多いだろうが、OpenSSL等で最も良く利用されるのはDER・PEM形式の鍵。
\end{frame}

\begin{frame}
\begin{center}
\begin{table}
\scriptsize
\caption{各データの表現形式の規定文書一覧}
\begin{tabular}{|c|c|c|c|}
\hline
& DER/PEM & SECG SEC1 & JWK/E/S \\
\hline
\hline
RSA公開鍵 & 1) PKCS\#1 & N/A & RFC7517 \\
 & 2)RFC5280\footnote[frame]{\scriptsize 証明書の中にあるSubjectPublicKeyInfoフィールド} & & \\
\hline
RSA秘密鍵 & RFC5958\footnote[frame]{\scriptsize 秘密鍵の暗号化サポート。RFC5958はPKCS\#8の拡張。}, PKCS\#1\footnote[frame]{\scriptsize RSA秘密鍵自体の構造を定義} & N/A & RFC7517 \\
\hline
ECC公開鍵 & RFC5480\footnote[frame]{\scriptsize 証明書の中にあるSubjectPublicKeyInfoフィールド} & SECG SEC1 v2 & RFC7517 \\
\hline
ECC秘密鍵 & RFC5958\footnote[frame]{\scriptsize 秘密鍵の暗号化をサポートするフォーマット}, RFC5915\footnote[frame]{\scriptsize ECC秘密鍵自体の構造を定義} & SECG SEC1 v2 & RFC7517 \\
\hline
(AES)共通鍵 & N/A & N/A & RFC7517 \\
\hline
公開鍵証明書 & RFC5280 (他) & N/A & N/A \footnote[frame]{\scriptsize X.509証明書へのURI等は記述可能 \url{https://tools.ietf.org/html/rfc7517}} \\
\hline
RSA暗号文 & N/A & N/A & RFC7516\\
\hline
ECDH+AES暗号文 & N/A & N/A & RFC7516\\
\hline
AES暗号文 & N/A & N/A & RFC7516 \\
\hline
HMAC & N/A & N/A & RFC7515\\
\hline
RSA署名 & N/A & N/A & RFC7515\\
\hline
ECDSA署名 & RFC5759 & N/A & RFC7515\\
\hline
\end{tabular}
\end{table}
\end{center}
\end{frame}


\begin{frame}
\begin{exampleblock}{\small WebCrypto API/Node.jsでサポートする公開鍵・秘密鍵フォーマット}
\begin{itemize}
 \item WebCrypto API: \\RFC5280形式 (公開鍵のみ)、RFC5958形式 (秘密鍵のみ)、JWK
 \item Node.js Crypto: \\RFC5280形式 (公開鍵のみ)、RFC5958形式 (秘密鍵のみ)、PKCS\#1形式
\end{itemize}
\end{exampleblock}
統一が取れていないため、環境に応じて適切な表現形式の変換が必要\footnote[frame]{\scriptsize jscuはこれらの鍵変換をサポート。\url{https://github.com/junkurihara/jscu/tree/develop/packages/js-crypto-key-utils}}
\end{frame}


\section{楕円曲線のパラメタについて}
\begin{frame}
\centering
{\Large 楕円曲線暗号のパラメタについて}
\end{frame}

\begin{frame}
\frametitle{ECDH, ECDSAの鍵選択(曲線の選択)}
\begin{block}{\small RSA暗号と楕円曲線暗号(ECDH/ECDSA)のパラメタ}
\begin{itemize}
 \item RSA: パラメタは「公開鍵ビット長」
 \item 楕円曲線: パラメタは「曲線の種類」
\end{itemize}
\end{block}
曲線の種類は、公開鍵ビット長を定めれば一意に決まるわけではない\footnote[frame]{\scriptsize e.g., 公開鍵長 256bitでは、P-256やP-256Kなど複数の利用可能な曲線の種類が存在。}。

\begin{alertblock}{}
※ただし、RSA暗号・楕円曲線暗号共々、``一般的には''\structure{公開鍵ビット長が大きいほど、より安全になり、より処理が重く}なるというトレードオフがある。 
\end{alertblock}
\end{frame}

\begin{frame}
では、どういった曲線を選べば良いのか?

\begin{block}{\small 代表的な楕円曲線パラメタの標準}
\begin{itemize}
 \item SEC2: 業界団体 SECG の標準\footnote[frame]{\scriptsize \url{http://www.secg.org/sec2-v2.pdf}}
 \item ANSI X9.62: 米国標準 \footnote[frame]{\scriptsize American National Standards Institute, ``Public Key Cryptography for the Financial Services Industry: The Elliptic Curve Digital Signature Algorithm (ECDSA),'' ANSI X9.62, November 2005.}
 \item NIST FIPS186-4: \footnote[frame]{\scriptsize \url{https://nvlpubs.nist.gov/nistpubs/FIPS/NIST.FIPS.186-4.pdf}}
\end{itemize}
\end{block}
それぞれで標準化されたパラメタは、大きくオーバーラップしている\footnote[frame]{\scriptsize \url{https://tools.ietf.org/html/rfc8422\#appendix-A}}が、\alert{最も頻繁に更新されている標準パラメタリストはNIST FIPS 186}のものである。
\end{frame}

\begin{frame}
\begin{block}{\small WebCrypto API/Node.js両方で選択できるNIST曲線パラメタ}
\begin{itemize}
 \item P-256: 公開鍵長 = 256bits, 安全性 $\simeq$ 128bit鍵AES
 \item P-384: 公開鍵長 = 384bits, 安全性 $\simeq$ 192bit鍵AES
 \item P-521: 公開鍵長 = 521bits, 安全性 $\simeq$ 256bit鍵AES
\end{itemize}
\end{block}
Bitcoin Blockchainで使用されている曲線\footnote[frame]{\scriptsize SECGパラメタ secp256k1あるいはP-256Kと呼ばれる。}は、WebCrypto APIでネイティブには未実装。
\end{frame}

\begin{frame}

曲線パラメタに対して、より学術的な脆弱性の有無の一覧:\\
Safe Curves: \url{https://safecurves.cr.yp.to/}

\begin{center}
 $\Downarrow$
\end{center}

\underline{現実的な攻撃かどうかはさておき}、\structure{NIST曲線パラメタには有効な攻撃が知られている}\footnote{\scriptsize \texttt{Safe?=False}の場合、攻撃が存在。}。

実装に不備がある場合に発生する脆弱性が多く、\alert{``正しく設計・実装されている既存ライブラリ''のAPI経由で利用する分には問題ない}と言ってもよいだろう。
\end{frame}




% %%%%%%%%%%%%%%%%%%%%%%%%%%%%%%%%%%%%%%%%%%%%%%%%%%%%%%%%%%%%%%%%%%%%%%%%%%%%%%%%%%%%%%%%%%%%%%%%%%%
% \backupbegin

% \section{Backup}
% \begin{frame}
%  以下バックアップ
% \end{frame}


% \begin{frame}
 
% \end{frame}

% \begin{frame}
% \frametitle{Appendix}
% This page is not counted.
% \end{frame}
% \backupend
\end{document}
%%%%%%%%%%%%%%%%%%%%%%%%%%%%%%%%%%%%%%%%%%%%%%%%%%%%%%%%%%%%%%%%%%%%%%%%%%%%%%%%%%%%%%%%%%%%%%%%%%%
%%%%%%%%%%%%%%%%%%%%%%%%%%%%%%%%%%%%%%%%%%%%%%%%%%%%%%%%%%%%%%%%%%%%%%%%%%%%%%%%%%%%%%%%%%%%%%%%%%%
%%%%%%%%%%%%%%%%%%%%%%%%%%%%%%%%%%%%%%%%%%%%%%%%%%%%%%%%%%%%%%%%%%%%%%%%%%%%%%%%%%%%%%%%%%%%%%%%%%%
%%%%%%%%%%%%%%%%%%%%%%%%%%%%%%%%%%%%%%%%%%%%%%%%%%%%%%%%%%%%%%%%%%%%%%%%%%%%%%%%%%%%%%%%%%%%%%%%%%%
%%%%%%%%%%%%%%%%%%%%%%%%%%%%%%%%%%%%%%%%%%%%%%%%%%%%%%%%%%%%%%%%%%%%%%%%%%%%%%%%%%%%%%%%%%%%%%%%%%%
%%%%%%%%%%%%%%%%%%%%%%%%%%%%%%%%%%%%%%%%%%%%%%%%%%%%%%%%%%%%%%%%%%%%%%%%%%%%%%%%%%%%%%%%%%%%%%%%%%%
